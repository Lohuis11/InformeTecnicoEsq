\section*{RESUMEN}

{\large El trabajo que a continuación se presenta, tiene como objetivo desarrollar un sistema web que se encargue de la parte administrativa de una institución de educación en este caso se implementara en la Escuela Secundaria Técnica No. 96 ubicada en la población de Santiago Yolomecatl, en la Mixteca del Estado de Oaxaca. Dirigida para los administrativos de la Institución, docentes, alumnos y personas interesadas en obtener información sobre la misma.}\\
\vspace{0.5cm}

{\large Para el desarrollo de este proyecto se utilizó la Metodología de Diseño de Hipermedia Orientad a Objetos(OOHDM por sus siglas en ingles), comenzando con los primeros pasos que indica la metodología se desarrollaron los diagramas UML, para especificar los usuarios que intervendrán, así como las funciones y acciones que cada uno podrá realizar dentro del sistema, los diagramas que se desarrollaron son: Casos de Uso, Diagramas de Secuencia, Diagrama de Clases y el Modelo Entidad-Relación para la Base de Datos, para el diseño y modelado de interfaces se realizo con la herramienta de Axure, además de hacer uso de PostgreSQL para el manejo de la Base de Datos donde se almacenará toda la información referente a la Institución, personal y alumnos que la integran. }\\
\vspace{0.5cm}

{\large El sistema final ayudara en la parte administrativa de la Institución, ya que permitirá registrar información de todos los estudiantes, docentes y personal administrativo que este inscrito a la misma, dentro del sistema se encontraran diferentes apartados para los diversos usuarios que interaccionen con él, por ejemplo contará con apartado para los estudiantes donde podrán ver sus calificaciones, su estatus de conducta entre otras cosas, un apartado para los profesores para que puedan subir las calificaciones de sus materias por mencionar alguna tarea que pueda realizar, el sistema también contara con un apartado para los administrativos, el área de prefectura, la biblioteca escolar, el centro de computo y para tener contabilizados los recursos materiales y financieros.}\\
\vspace{0.5cm}

{\large Finalmente el sistema estará diseñado para que sea amable con el usuario y de fácil uso para el mismo.}\\


