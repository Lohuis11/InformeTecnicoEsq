\renewcommand{\thesection}{\Roman{section}}
\section{Marco Teórico}

\renewcommand{\thesection}{\arabic{section}}
\subsection{Marco Conceptual}

{\large Dentro del área de la programación existen diversas palabras que son especificas de la actividad y que para algunas personas que no estan familiarizados con esta practica se les dificulta poder entender ciertas palabras, debido a esto a continuación se presentan algunos conceptos de mas relevancia para el mejor entendimiento de este informe}\\

\begin{itemize}
\item SERVIDOR
\end{itemize}
{\large Almacena de forma organizada la estructura de la información del sitio web para servir los contenidos en relación a las peticiones del navegador.}\\

\begin{itemize}
\item TECNOLOGÍAS DE CLIENTE
\end{itemize}
{\large Son aquellas que permiten crear interfaces de usuario y establecer comunicación con el servidor basadas en HTML, CSS y JavaScript, en este caso, el navegador actúa como intérprete.}\\

\begin{itemize}
\item TECNOLOGÍAS DE SERVIDOR
\end{itemize}
{\large Permiten implementar comportamientos de la aplicación web en el servidor, los lenguajes de programación más utilizados son Java EE, .NET, PHP, Ruby on Rails, Python, Django, Groovy, Node.js, entre otros.}\\

\begin{itemize}
\item SCRIPTING
\end{itemize}
{\large Gracias al scripting las páginas pueden programarse con distintos lenguajes de script, aunque principalmente se utiliza JavaScript, que modifica la página gracias a su capacidad de ejecutar código cuando se interactúa con ella. JavaScript inicialmente era un lenguaje interpretado pero actualmente también se ejecuta con máquinas virtuales en los navegadores aumentando la velocidad de ejecución y eficiencia de memoria. Es de tipado dinámico y funcionalmente orientado a objetos (basado en prototipos).
Existen multitud de bibliotecas (APIS) para el desarrollo web y de aplicaciones, pero las más utilizadas son JQuery y Underscore.js.
}\\

\begin{itemize}
\item AJAX
\end{itemize}
{\large Acrónimo de Asynchronous JavaScript And XML (JavaScript asíncrono y XML), es una técnica de desarrollo web para crear aplicaciones interactivas o RIA (Rich Internet Applications). Estas aplicaciones se ejecutan en el cliente, es decir, en el navegador de los usuarios mientras se mantiene la comunicación asíncrona con el servidor en segundo plano. De esta forma es posible realizar cambios sobre las páginas sin necesidad de recargarlas, lo que significa aumentar la interactividad, velocidad y usabilidad en las aplicaciones.}\\

\begin{itemize}
\item IDE
\end{itemize}
{\large Entorno de desarrollo integrado (integrated development environment). Es un entorno de programación que ha sido empaquetado como un programa de aplicación, es decir, consiste en un editor de código, un compilador, un depurador y un constructor de interfaz gráfica (GUI). Los IDEs pueden ser aplicaciones por sí solas o pueden ser parte de aplicaciones existentes.}\\

\begin{itemize}
\item FRAMEWORK
\end{itemize}
{\large Infraestructura digital, es una estructura conceptual y tecnológica de soporte definido, normalmente con artefactos o módulos de software concretos, con base a la cual otro proyecto de software puede ser más fácilmente organizado y desarrollado. Típicamente, puede incluir soporte de programas, bibliotecas, y un lenguaje interpretado, entre otras herramientas, para así ayudar a desarrollar y unir los diferentes componentes de un proyecto.}\\

\subsection{Discusión Teórica}

{\large La escuela secundaria tecnica No. 96 esta ubicada en la población de Stgo. Yolomecatl}\\