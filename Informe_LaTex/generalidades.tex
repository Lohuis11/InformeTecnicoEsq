\renewcommand{\thesection}{\Roman{section}}
\section{Generalidades del Proyecto}

\renewcommand{\thesection}{\arabic{section}}
\subsection{Introducción}

{\large En la actualidad el desarrollo de procesos se ha ido optimizando cada vez más, gracias al gran avance de la tecnología y la implementación que esta misma se hace, hoy en día ya es común ver diversidad de procesos automatizados o en donde hace intervención una computadora.}\\

{\large Aun con todos los avances que ha hecho la tecnología hay lugares donde aun se realizan procesos de forma manual, tal es el caso de la Escuela Secundaria Tecnica No. 96 que se encuentra ubicada en la población de Santiago Yolomecatl, perteneciente a la región Mixteca del Estado de Oaxaca.}\\

{\large En esta institución de educación publica aun no cuentan con un sistema digital para poder realizar sus diversos procesos administrativos que actualmente realizan por medio de hojas físicas y de forma personal, pudiendo generar algún tipo de incoveniente al momento de hacer los tramites.}\\

{\large Es por ello que se escogio esta dependencia para realizar la residencia profesional, aplicando los conocimientos adquiridos durante el tiempo que se cursaron los semestres, escogiendo la mejor solución para el problema que presenta la dependencia. }\\

{\large Despues de realizar la entrevista y de saber en realidad los proesos que el personal de la dependencia realiza se tuvo una mejor visión del problema presentado, para esto se determino que la mejor solución es realizar un control escolar por el cual el personal, alumnos y demás personas interesadas en la institución puedan realzar sus tramites.}\\

{\large Para el desarrollo del control escolar se tomaron en cuenta diversas herramientas con las que se pudiera realizar el proyecto, tomando como la mejor y mas viable hacer el desarrollo de un sistema mediante tecnologías web, haciendo uso de Angular como framework para el desarrollo.}\\

\subsection{Descripción de la empresa u organización y del puesto o área del trabajo el estudiante}

\subsubsection{Descripcion de la Empresa}

{\large La educación secundaria es el tercero y último nivel que conforma a la educación básica. Se cursa en tres grados, tiene como antecedente obligatorio la educación primaria y es de carácter propedéutica, es decir, necesaria para ingresar al nivel medio superior.}\\

{\large En la modalidad de secundaria técnica, la enseñanza que se imparte incluye las materias académicas de educación secundaria general, además de asignaturas para capacitar a los educandos en actividades tecnológicas industriales, comerciales, agropecuarias, pesqueras y forestales; su fin es preparar al alumno para que ingrese al nivel medio superior y, además, darle la oportunidad de incorporarse al mercado de trabajo con una educación tecnológica de carácter propedéutico.}\\

{\large La Escuela Secundaria Técnica No. 96 ubicada en la población de Santiago Yolomecatl, Teposcolula, Oaxaca. Pertence a la educación básica en la formación de los jóvenes que se dedican al estudio, forma parte de una amplia red de Escuelas Secundarias Tecnicas ubicadas en diferentes puntos para que los jóvenes puedan realizar sus estudios de este nivel. }\\

{\large Conformada por una matricula estudiantil de 300 alumnos, una plantilla de 20 docentes, un grupo de 10 personas que conforman el área administrativa y dirigidos por el Maestro Panuncio López Chavez Director General de la Institución.}\\

\paragraph{Misión}

{\large La escuela secundaria tecnica No. 96 esta ubicada en la población de Stgo. Yolomecatl}\\

\paragraph{Visión}

{\large La escuela secundaria tecnica No. 96 esta ubicada en la población de Stgo. Yolomecatl}\\

\paragraph{Objetivos}

{\large La escuela secundaria tecnica No. 96 esta ubicada en la población de Stgo. Yolomecatl}\\

\paragraph{Organigrama}

{\large La escuela secundaria tecnica No. 96 esta ubicada en la población de Stgo. Yolomecatl}\\

\subsubsection{Descripción del área de trabajo en donde se inserta el proyecto de residencia}

{\large La escuela secundaria tecnica No. 96 esta ubicada en la población de Stgo. Yolomecatl}\\

\subsection{Problemas a resolver}

{\large En esta Institución existen diversos problemas que se encuentran presentes en las diferentes áreas que la integran, pudiendo resaltar algunos como mas prioritarios que otros.}\\

\begin{itemize}
\item Agilización de Tramites
\end{itemize}
{\large Dentro de la institución se realizan diversos tramites en diferentes áreas, por ejemplo el área de administración requiere de inscribir alumnos, generar boletas de calificaciones entre otros tramites que actualmente los realizan de forma física, utilizando formatos en hojas y se llevan bastante tiempo para realizar estos trámites, en otras áreas como la de contraloría y la de biblioteca cada determinado tiempo se les es requerido generar un reporte donde especifiquen los movimientos que realizaron así como dar a conocer los recursos con los que cuentan es por ello que la implementación del sistema les permitirá agilizar la generación de estos trámites.}\\

\begin{itemize}
\item Ahorro de Tiempo
\end{itemize}
{\large Un problema que aqueja a la mayoría del personal administrativo y docente que forma parte de la Escuela Secundaria es el tiempo que llegan a ocupar para realizar ciertas actividades que les competen, por ejemplo los docentes tienden a ocupar varias horas al momento de generar sus listas de calificaciones, el sistema que se implementara permitirá que generen sus listas de forma digital, para que ahorren tiempo en rellenar el formato que les compete para esa actividad. }\\

\begin{itemize}
\item Control de Información
\end{itemize}
{\large Debido a que actualmente los tramites que se realizan en la institución se hacen de forma física, hay ocasiones que se tiene un mal manejo de la información, por ejemplo el área de contraloría que es la encargada de llevar el control de los recursos materiales y financieros que existen en la institución a veces se les complica un poco tener contabilizados los materiales, esto debido a que en ocasiones llegan mas de una persona a solicitar algún material y en su momento no pueden atender a todos simultáneamente, y provoca un error al momento de contabilizar lo prestado, es por ello que con el sistema bastara con un par de click’s para marcar a quien y cual fue el material prestado.}\\

\begin{itemize}
\item Presentación de la Institución.
\end{itemize}
{\large Un punto importante al momento de promocionar a la institución es como se presenta a la misma y el medio por el cual se presenta, el sistema a parte de ayudar en la administración permitirá presentar información relacionada a la Institución, como sus instalaciones, el plantel docente, administrativo y manual que la integran y otras cosas de importancia como la ubicación o contacto con la misma.}\\

\subsection{Objetivos}

\subsubsection{Objetivo General}

{\large Desarrollar un sistema web para el control escolar, material y bibliotecario de la Escuela Secundaria Técnica no.96 }\\

\subsubsection{Objetivos Especificos}

\begin{itemize}
\item Recaudar información sobre el funcionamiento actual de los procesos académicos, mediante entrevistas.
\item Realizar un análisis de la información obtenida.
\item Realizar un análisis de las necesidades para la operatividad del sistema.
\item Realizar el diseño conceptual.
\item Realizar el diseño navegacional.
\item Realizar el diseño de Interfaz Abstracta.
\item Realizar la codificación de los módulos.
\item Realizar pruebas del sistema.
\item Realizar la implementación.     
\end{itemize}

\subsection{Justificacion}

{\large El proyecto se origina por la necesidad de no contar con un control escolar, ya que actualmente se hace de manera tradicional (expedientes físicos) corriendo el riesgo del mal manejo del mismo, además de realizar procesos laboriosos ya que para su elaboración de una boleta de calificaciones se tiene que llevar a cabo diferentes fases, empezando por el listado de calificaciones por parte de los docentes, después se hace una revisión por los administrativos de la institución, por lo que es conveniente y necesario utilizar las tecnologías actuales para mejorar este proceso.}
\vspace{0.5cm}
{\large En cuanto al inventario de recursos materiales, no se cuenta con el mismo, haciendo una tarea complicada saber con qué material cuenta la institución y en qué estado puede estar dicho material; en cuanto al control bibliotecario es necesario ya que en muchos casos se desconoce de todo el material bibliotecario con el que se cuenta y el estatus de cada elemento que lo integra.}
\vspace{0.5cm}
{\large Por último en cuanto al seguimiento de los alumnos se realiza de manera tradicional haciendo un poco complicado la comunicación con los tutores de los alumnos debido a sus actividades que desarrollan por lo tanto es una buena alternativa el llevar a cabo una sección de seguimiento para los alumnos para que los tutores puedan estar al tanto de la situación de sus hijos. Por lo que se pretende con este proyecto optimizar todos estos procesos y recursos con los que cuenta la institución. }\\